% Maths symbols, equation environments and physics macros:
\usepackage{amsmath,amssymb}
\usepackage{physics}

% Graphics:
\usepackage{graphicx}

% For an mbox command that works in math mode (i.e that contains math):
\usepackage{mathtools}

% Colours:
\usepackage[table]{xcolor}
\definecolor{pastelred}{RGB}{200,80,80}
\definecolor{pastelgreen}{RGB}{0,150,80}
\definecolor{battleshipgrey}{rgb}{0.52, 0.52, 0.51}


% Fonts:
\usepackage[
  % Uppercase Greek letters should be italic by default:
  math-style=ISO,
  % Uppercase bold symbols should also be italic unless set otherwise:
  bold-style=ISO,
  % Upright \nabla though:
  nabla=upright,
  % \mathrm,\mathit,\mathbf and operators should use the maths font, not the text font:
  mathup=sym,mathit=sym,mathbf=sym
]{unicode-math}

% Garamond for normal text:
\setmainfont[
    Mapping=tex-text,
    Contextuals=Alternate,
    Numbers={OldStyle,Proportional},
    BoldFont={GaramondPremrPro-Bd},
    ItalicFont={GaramondPremrPro-It},
    BoldItalicFont={GaramondPremrPro-BdIt}
    ]{Garamond Premier Pro}

% Monospaced font:
\setmonofont{Ubuntu Mono}

% Asana for all non-alphanumeric symbols, operators, etc:
\setmathfont{Asana Math}

% Latin Modern for integrals, sums, \nabla, and other symbold determined to be missing from Minion
% and Asana:
\setmathfont[range={\mathop,\nabla,\triangleright}]{Latin Modern Math}


% Minion pro for alphanumeric symbols, inlcuding Greek:
\setmathfont{MinionPro}[
    range={up}/{latin,Latin,num,Greek,greek},
    SizeFeatures = {
        {Size = -8.41, Font = MinionPro-Capt},
        {Size = 8.41-13.01, Font = MinionPro-Regular},
        {Size = 13.01-19.91, Font = MinionPro-Subh},
        {Size = 19.91-, Font = MinionPro-Disp}
    }]
\setmathfont{MinionPro-It}[
    range={it}/{latin,Latin,num,Greek,greek},
    SizeFeatures = {
        {Size = -8.41, Font = MinionPro-ItCapt},
        {Size = 8.41-13.01, Font = MinionPro-It},
        {Size = 13.01-19.91, Font = MinionPro-ItSubh},
        {Size = 19.91-, Font = MinionPro-ItDisp}
    }]
\setmathfont{MinionPro-Bold}[
    range={bfup}/{latin,Latin,num,Greek,greek},
    SizeFeatures = {
        {Size = -8.41, Font = MinionPro-BoldCapt},
        {Size = 8.41-13.01, Font = MinionPro-Bold},
        {Size = 13.01-19.91, Font = MinionPro-BoldSubh},
        {Size = 19.91-, Font = MinionPro-BoldDisp}
    }]
\setmathfont{MinionPro-BoldIt}[
    range={bfit}/{latin,Latin,num,Greek,greek},
    SizeFeatures = {
        {Size = -8.41, Font = MinionPro-BoldItCapt},
        {Size = 8.41-13.01, Font = MinionPro-BoldIt},
        {Size = 13.01-19.91, Font = MinionPro-BoldItSubh},
        {Size = 19.91-, Font = MinionPro-BoldItDisp}
    }]

% Use math font for blackboard bold:
\DeclareMathAlphabet{\mathbb}{U}{msb}{m}{n}

% Use swashed Minion Pro for mathcal:
% \setmathfont[range=cal,Contextuals=Swash]{MinionPro-It}

% Workaround for bug where the above commented out line doesn't work - it works in xelatex and will
% presumably be fixed for lualatex in the future. In the meantime we define a text font face that
% uses swashed Minion Pro, and redefine the \mathcal command to use it:
\newfontface\swash[Contextuals=Swash]{MinionPro-It}
\renewcommand\mathcal[1]{\mathmbox{\textrm{\swash #1}}}

% redefine \texttt to be coloured:
\renewcommand\texttt[1]{{\ttfamily\color{pastelgreen}#1}}

% Typesetting tweaks for pedants and perfectionists:
\usepackage{microtype}

% Better behaved \cite command:
\usepackage{cite}

% For calculating text sizes
\usepackage{calc}

% Allow rotating floats - useful for landscape float pages:
\usepackage{rotating}

% Make commands run at the start of a new page. Useful for changing the margins on a float page
% without having to manually start a new page and thus have the rest of the current page
% potentially blank:
\usepackage{afterpage}

% Drop caps:
\usepackage{lettrine}

% For making fake blocks of text for testing:
\usepackage{lipsum}

% For changing page margins, useful used in combination with \afterpage for a float page:
\usepackage{geometry}

% Captions on the side of figures
\usepackage{subfig}
\captionsetup*[figure]{position=top}

% Source code listings:
\usepackage{minted}
  % % A command for including Python source from a file, like: \python{myfile.py}
  % \newcommand{\python}[1]{\spacing{0.7}\inputminted[linenos, numbersep=5pt, frame=lines, framesep=2mm,fontsize=\scriptsize]{python}{#1}}

  % A command for including Python source from a file, like: \python{myfile.py}
  \newcommand{\python}[1]{\inputminted[linenos, numbersep=5pt, frame=lines, framesep=2mm,fontsize=\scriptsize]{python}{#1}}

% For verbatim text input
\usepackage{verbatim}

% For typesetting algorithms:
\usepackage{algorithm,algpseudocode,float}

% For stacking things:
\usepackage{stackengine}

% Page-breakable algorithm environment
\makeatletter
\newenvironment{breakablealgorithm}
  {% \begin{breakablealgorithm}
   \begin{center}
     \refstepcounter{algorithm}% New algorithm
     \hrule height.8pt depth0pt \kern2pt% \@fs@pre for \@fs@ruled
     \renewcommand{\caption}[2][\relax]{% Make a new \caption
       {\raggedright\textbf{\ALG@name~\thealgorithm} ##2\par}%
       \ifx\relax##1\relax % #1 is \relax
         \addcontentsline{loa}{algorithm}{\protect\numberline{\thealgorithm}##2}%
       \else % #1 is not \relax
         \addcontentsline{loa}{algorithm}{\protect\numberline{\thealgorithm}##1}%
       \fi
       \kern2pt\hrule\kern2pt
     }
  }{% \end{breakablealgorithm}
     \kern2pt\hrule\relax% \@fs@post for \@fs@ruled
   \end{center}
  }
\makeatother


% Hyperlinked references, contents:
\usepackage[bookmarksopen,
  pagebackref,
  pdfpagelayout=TwoPageRight,
  colorlinks=true,
  urlcolor=pastelred,
  citecolor=pastelred,
  filecolor=pastelred,
  linkcolor=pastelred,
  linktocpage=true]
  {hyperref}

% Hyperlinked back-references in the bibliography
\renewcommand*{\backrefalt}[4]{
  \ifnum#1=1
    [p~#2]
  \else
    [pp~#2]
  \fi
}

% Make links to figures link to the top of the figure instead of the caption
\usepackage[all]{hypcap}

% Remove need for extra pass of latex to make bookmarks:
\usepackage{bookmark}

% Links to DOIs in bibliography
\usepackage{doi}

% Provides \RaggedLeft and \RaggedRight commands for ragged text with occasional hyphenation:
\usepackage{ragged2e}

% Import hyperlinks from .pax files when including external pdfs

% Hacks to make pax work with lualatex, from
% https://tex.stackexchange.com/questions/60201/getting-pax-pdfpages-to-work-with-xelatex
\makeatletter
\let\pdfescapename=\pdf@escapename
\let\pdfstrcmp=\pdf@strcmp
\makeatother
\usepackage{pax}

% For including external PDF documents as pages in output:
\usepackage[final]{pdfpages}

% For defining abbreviations (that will be written in full their first use)
% \usepackage{glossaries}
% \glsdisablehyper
% \setacronymstyle{long-sc-short}

% For making some pages landscape in the PDF
\usepackage{pdflscape}

% For multi-row tables:
\usepackage{hhline}
\usepackage{multirow}

% Tick and cross mark
\usepackage{pifont}% http://ctan.org/pkg/pifont
\newcommand{\cmark}{\ding{51}}%
\newcommand{\xmark}{\ding{55}}%

