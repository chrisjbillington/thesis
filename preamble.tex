\documentclass[a4paper,10pt,twoside]{memoir}

% Maths symbols, equation environments and physics macros:
\usepackage{amsmath,amssymb}
\usepackage{physics}

% For an mbox command that works in math mode (i.e that contains math):
\usepackage{mathtools}

% Non-italic greek characters, mostly good for \upmu in micrometres, microseconds etc:
\usepackage{upgreek}

% Colours:
\usepackage{xcolor}
\definecolor{pastelred}{RGB}{200,80,80}
\definecolor{pastelgreen}{RGB}{0,150,80}


% Fonts:
\usepackage[
  % Uppercase Greek letters should be italic by default:
  math-style=ISO,
  % Uppercase bold symbols should also be italic unless set otherwise:
  bold-style=ISO,
  % \mathrm,\mathit,\mathbf and operators should use the maths font, not the text font:
  mathup=sym,mathit=sym,mathbf=sym
]{unicode-math}

% Garamond for normal text:
\setmainfont[
    Mapping=tex-text,
    Contextuals=Alternate,
    Numbers={OldStyle,Proportional},
    BoldFont={GaramondPremrPro-Bd},
    ItalicFont={GaramondPremrPro-It},
    BoldItalicFont={GaramondPremrPro-BdIt}
    ]{Garamond Premier Pro}

% Monospaced font:
\setmonofont{Ubuntu Mono}
% redefine \texttt to be coloured:
\renewcommand\texttt[1]{{\ttfamily\color{pastelgreen}#1}}

% Caption name/title style:
\captionnamefont{\normalfont\small\itshape}
\captiontitlefont{\small\rmfamily}

% Asana for all non-alphanumeric symbols, operators, etc:
\setmathfont{Asana Math}

% Latin Modern for integrals, sums etc:
\setmathfont[range={\mathop}]{Latin Modern Math}

% Minion pro for alphanumeric symbols, inlcuding Greek:
\setmathfont{Minion Pro}[
    range={up}/{latin,Latin,num,Greek,greek},
    SizeFeatures = {
        {Size = -8.41, Font = MinionPro-Capt},
        {Size = 8.41-13.01, Font = MinionPro-Regular},
        {Size = 13.01-19.91, Font = MinionPro-Subh},
        {Size = 19.91-, Font = MinionPro-Disp}
    }]
\setmathfont{Minion Pro Italic}[
    range={it}/{latin,Latin,num,Greek,greek},
    SizeFeatures = {
        {Size = -8.41, Font = MinionPro-ItCapt},
        {Size = 8.41-13.01, Font = MinionPro-It},
        {Size = 13.01-19.91, Font = MinionPro-ItSubh},
        {Size = 19.91-, Font = MinionPro-ItDisp}
    }]
\setmathfont{Minion Pro Bold}[
    range={bfup}/{latin,Latin,num,Greek,greek},
    SizeFeatures = {
        {Size = -8.41, Font = MinionPro-BoldCapt},
        {Size = 8.41-13.01, Font = MinionPro-Bold},
        {Size = 13.01-19.91, Font = MinionPro-BoldSubh},
        {Size = 19.91-, Font = MinionPro-BoldDisp}
    }]
\setmathfont{Minion Pro Bold Italic}[
    range={bfit}/{latin,Latin,num,Greek,greek},
    SizeFeatures = {
        {Size = -8.41, Font = MinionPro-BoldItCapt},
        {Size = 8.41-13.01, Font = MinionPro-BoldIt},
        {Size = 13.01-19.91, Font = MinionPro-BoldItSubh},
        {Size = 19.91-, Font = MinionPro-BoldItDisp}
    }]


% Make mathcal use sensible symbols - can't seem to get lualatex to use Minion or Garamond swash
% letters. Works in xelatex, but presently that renders Minion Pro with unicode-math pretty badly.
\usepackage{calrsfs}
\DeclareMathAlphabet{\mathcal}{OMS}{zplm}{m}{n}

% Typesetting tweaks for pedants and perfectionists:
\usepackage{microtype}

% Better behaved \cite command:
\usepackage{cite}

% Allow rotating floats - useful for landscape float pages:
% \usepackage[xetex]{rotating}
\usepackage{rotating}

% Graphics:
% \usepackage[xetex]{graphicx}
\usepackage{graphicx}
% \usepackage[xetex,draft]{graphicx} %uncomment to compile in draft mode - no images, faster

% Make commands run at the start of a new page. Useful for changing the margins on a float page
% without having to manually start a new page and thus have the rest of the current page
% potentially blank:
\usepackage{afterpage}

% Drop caps:
\usepackage{lettrine}

% For making fake blocks of text for testing:
\usepackage{lipsum}

% For changing page margins, useful used in combination with \afterpage for a float page:
\usepackage{geometry}

% Source code listings:
\usepackage{minted}
  % A command for including Python source from a file, like: \python{myfile.py}
  \newcommand{\python}[1]{\spacing{0.7}\inputminted[linenos, numbersep=5pt, frame=lines, framesep=2mm,fontsize=\scriptsize]{python}{#1}}

% For typesetting algorithms:
\usepackage{algpseudocode}

% Hyperlinked references, contents:
\usepackage[bookmarksopen,
  pagebackref,
  colorlinks=true,
  urlcolor=pastelred,
  citecolor=pastelred,
  filecolor=pastelred,
  linkcolor=pastelred,
  linktocpage=true]
  {hyperref}

% Hyperlinked back-references in the bibliography
\renewcommand*{\backrefalt}[4]{
  \ifnum#1=1
    [p~#2]
  \else
    [pp~#2]
  \fi
}

% Links to DOIs in bibliography
\usepackage{doi}


% This page left intentionally blank:
  \makeatletter \def\clearforchapter{
    \clearpage\if@twoside \ifodd\c@page\else \hbox{}\vfill
    \begin{center}This page intentionally left blank
    \end{center}\vfill \thispagestyle{cleared}%
  \newpage\if@twocolumn\hbox{}\newpage\fi\fi\fi}\makeatother

  \newcommand*{\blankpage}{%
  \vspace*{\fill}
  \centering This page intentionally left blank
  \vspace{\fill}}
  \makeatletter
  \renewcommand*{\cleardoublepage}{\clearpage\if@twoside \ifodd\c@page\else
  \blankpage
  \thispagestyle{empty}
  \newpage
  \if@twocolumn\hbox{}\newpage\fi\fi\fi}
  \makeatother


% Make contents go deep
\setcounter{secnumdepth}{5}
\setcounter{tocdepth}{5}


%% Make footnotes go in the margin %%
\footnotesinmargin
\setlength{\footmarkwidth}{0em}
\setlength{\footmarksep}{0em}
\setlength{\footparindent}{0em}

\usepackage{ragged2e}

\renewcommand{\foottextfont}{\footnotesize\raggedright}

\usepackage{marginfix}

%%%%%%%%%%%%%%%%%%%%%%%%%%%%%%%%%%%%%%%
%%%% Defining custom page layout %%%%%%
\settrims{0pt}{0pt}
\settypeblocksize{220.8mm}{4.5in}{*}
\setlrmargins{25.4mm}{*}{*}
\setulmargins{1.5in}{*}{*}
\setmarginnotes{5mm}{40mm}{\onelineskip}
\setheadfoot{\onelineskip}{2\onelineskip}
\setheaderspaces{*}{2\onelineskip}{*}

\checkandfixthelayout
%%%%%%%%%%%%%%%%%%%%%%%%%%%%%%%%%%%%%%%

% Macros:

% \hat: QM Operator
\let\oldhat\hat
\renewcommand{\hat}[1]{\skew{1.5}\oldhat{\mathmbox{#1}}}

% \vec: Bold, italic vector
\renewcommand{\vec}[1]{\symbfit{#1}}

% \upvec: Non-italic vector
\newcommand{\upvec}[1]{\symbfit{#1}}

% Bras, kets, brakets, ketbras, expectation values, and matrix elements. Versions that don't resize
% start with a lowercase letter, versions that will resize start with uppercase letter:

\let\oldbra\bra
\renewcommand\bra[1]{\oldbra*{#1}}
\newcommand\Bra[1]{\oldbra{#1}}

\let\oldket\ket
\renewcommand\ket[1]{\oldket*{#1}}
\newcommand\Ket[1]{\oldket{#1}}

\let\oldbraket\braket
\renewcommand\braket[2]{\oldbraket*{#1}{#2}}
\newcommand\Braket[2]{\oldbraket{#1}{#2}}

\let\oldketbra\ketbra
\renewcommand\ketbra[2]{\oldketbra*{#1}{#2}}
\newcommand\Ketbra[2]{\oldketbra{#1}{#2}}

\let\oldev\ev
\renewcommand\ev[1]{\oldev*{#1}}
\newcommand\Ev[1]{\oldev{#1}}

\let\oldmatrixel\matrixel
\renewcommand\matrixel[3]{\oldmatrixel*{#1}{#2}{#3}}
\newcommand\Matrixel[3]{\oldmatrixel{#1}{#2}{#3}}

% Euler's number, shouldn't be italic:
\def\e{\mathrm{e}}

