% Skew accents that go over characters to make them more centered:

% \hat:
\let\oldhat\hat
\renewcommand{\hat}[1]{\skew{1.5}\oldhat{\mathmbox{#1}}}

% \tilde:
\let\oldtilde\tilde
\renewcommand{\tilde}[1]{\skew{1.5}\oldtilde{\mathmbox{#1}}}

% \dot:
\let\olddot\dot
\renewcommand{\dot}[1]{\skew{1.5}\olddot{\mathmbox{#1}}}

% \vec: Bold vector. Italic or not depending on unicode-math options.
\renewcommand{\vec}[1]{\symbf{#1}}

% Bras, kets, brakets, ketbras, expectation values, and matrix elements. Versions that don't resize
% start with a lowercase letter, versions that will resize start with uppercase letter:

\let\oldabs\abs
\renewcommand\abs[1]{\oldabs*{#1}}
\newcommand\Abs[1]{\oldabs{#1}}

\let\oldbra\bra
\renewcommand\bra[1]{\oldbra*{#1}}
\newcommand\Bra[1]{\oldbra{#1}}

\let\oldket\ket
\renewcommand\ket[1]{\oldket*{#1}}
\newcommand\Ket[1]{\oldket{#1}}

\let\oldbraket\braket
\renewcommand\braket[2]{\oldbraket*{#1}{#2}}
\newcommand\Braket[2]{\oldbraket{#1}{#2}}

\let\oldketbra\ketbra
\renewcommand\ketbra[2]{\oldketbra*{#1}{#2}}
\newcommand\Ketbra[2]{\oldketbra{#1}{#2}}

\let\oldev\ev
\renewcommand\ev[1]{\oldev*{#1}}
\newcommand\Ev[1]{\oldev{#1}}

\let\oldmatrixel\matrixel
\renewcommand\matrixel[3]{\oldmatrixel*{#1}{#2}{#3}}
\newcommand\Matrixel[3]{\oldmatrixel{#1}{#2}{#3}}

% A placeholder for zeros in matrices - a \cdot (but with a smaller command so that things are
% neater in latex source); first \makebox sets the width, second zero-width \makebox contains a
% phantom that sets the height:
\newcommand\mb{\phantom{0}}
% \newcommand\md{\makebox[0pt]{$\cdot$}\phantom{0}}
\newcommand\md{\makebox[\widthof{$0$}][c]{$\cdot$}}

% A smallmatrix with even smaller column separation:
\newenvironment{xsmallmatrix}
    {\renewcommand\thickspace{\kern1pt}\smallmatrix}
    {\endsmallmatrix}

% Imaginary and real parts as normal operators instead of script font:
\DeclareMathOperator{\im}{Im}
\DeclareMathOperator{\re}{Re}

% Sign function:
\DeclareMathOperator{\sgn}{sgn}

% Two argument arctan function:
\DeclareMathOperator{\arctantwo}{arctan2}

% Principle argument of complex number:
\DeclareMathOperator{\Arg}{Arg}
\DeclareMathOperator{\fft}{\textsc{fft}}

% % Minimum:
% \DeclareMathOperator{\min}{min}

% % Maximum:
% \DeclareMathOperator{\max}{max}

% A figref command to standardise the way figures are referenced:
\newcommand\figref[1]{Figure \ref{#1}}

% Similarly, a table command:
\newcommand\tableref[1]{Table\ref{#1}}

% Units
\newcommand\unit[1]{\,\mathrm{#1}}

% Scientific notation: times-ten-to-the
\newcommand\E[1]{\times 10^{#1}}

% \mathup in braces:
\newcommand\up[1]{{\mathup{#1}}}

% Imaginary unit, upright:
\newcommand\ii{\up{i}}

% Euler's number, upright:
\newcommand\ee{\up{e}}

% Big O notation. Choose mathscr instead of mathcal because Minion's swashed O doesn't look any different to its regular O.
\newcommand\Ord[1]{\mathscr{O}(#1)}
\newcommand\Ordx[1]{\mathscr{O}\left(#1\right)}

% Abbreviations:
% \newacronym{rk4}{rk4}{fourth order Runge--Kutta}
% \newacronym{rk4ip}{rk4ip}{fourth order Runge--Kutta in the interaction picture}
% \newacronym{rk4ilip}{rk4ilip}{fourth order Runge--Kutta in an instantaneous local interaction picture}
% \newacronym{sp}{sp}{Schr\"odinger picture}
% \newacronym{ip}{ip}{interaction picture}
% \newacronym{de}{de}{differential equation}

\newcommand\runmanager{\texttt{runmanager}}
\newcommand\labscript{\texttt{labscript}}
\newcommand\blacs{\texttt{BLACS}}
\newcommand\lyse{\texttt{lyse}}
\newcommand\bias{\texttt{BIAS}}
\newcommand\runviewer{\texttt{runviewer}}
\newcommand\bec{{\scshape bec}}
\newcommand\mot{{\scshape mot}}
\newcommand\http{{\scshape http}}
\newcommand\hdf{{\scshape hdf5}}
\newcommand\ac{{\scshape ac}}
\newcommand\rf{{\scshape rf}}
\newcommand\api{{\scshape api}}
\newcommand\dds{{\scshape dds}}
