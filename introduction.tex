\chapter{Introduction}\label{chap:introduction}

\lettrine[lines=3]{T}{he subject of study} of this thesis is Bose--Einstein condensation, as well as associated experimental and theoeretical techniques and phenomena in cold atom physics. The following chapters describe my work in a cold atom research group over the past several years, pertaining to apparatus construction, experiment, theory, and software design and development. An overarching theme is \emph{state-dependent forces} on cold atoms. Selectively subjecting atoms to forces based on what state they are in is at the core of many phenomena in cold atom physics. As I go into in the following chapters, different types of state selectivity allow for cooling and imaging techniques that would otherwise not be possible, momentum state-selectivity is central to wave-mixing phenomena; and semiclassical models run into a problem when state-selective forces cannot be disregarded in determining the classical force that atoms modelled semiclasically ought to be subjected to.

Bose--Einstein condensates (\textsc{bec}s) in dilute atomic gases are superfluids that can be created in the lab at extremely low temperatures. This strange state of matter was predicted in 1925 by Bose and Einstein \cite{bose_plancks_1924, einstein_quantentheorie_1925} , first produced experimentally in 1995 \cite{anderson_observation_1995} in a cloud of rubidium atoms, and has since been made out of many other atoms, usually alkali metals \cite{davis_bose-einstein_1995, modugno_bose-einstein_2001, bradley_bose-einstein_1997, weber_bose-einstein_2003}. In a \textsc{bec}, a macroscopic sample of bosonic atoms all occupy the same quantum state, and many of the features of the single particle wavefunctions are exhibited by the cloud as a whole. Bose--Einstein condensation and cold atoms and ions more generally have rich applications in precision measurement \cite{robins_atom_2013, cronin_optics_2009}, quantum computation \cite{ladd_quantum_2010, negretti_quantum_2011} and quantum simulation \cite{bloch_quantum_2012, blatt_quantum_2012}.  

\section{Chapter overview}

Various experimental techniques are used to produce and study Bose--Einstein condensates, many of which exploit of necessitate an understanding of the quantum behaviour of the atomic systems in question. I summarise some of these techniques and detail the atomic physics principles underlying them in chapter \ref{chap:atomic_physics}.

The fields of Bose--Einstein condensation and cold atoms more generally enjoy a tight coupling between theory and experiment, not least because of the enduring usefulness and accuracy of mean-field theory. In mean-field theory, the quantum matter field operator of the atoms comprising a Bose--Einstein condensate is replaced with its expectation value at each point in space, allowing the entire multi-particle system to be modelled with little more computational complexity than that required to model a single-particle wavefunction.\footnote{Mean field theory is accurate in the low-temperature limit, in which it has some remaining limitations---it does not for example predict the observed $s$-wave scattering halos when two \textsc{bec} wavepackets collide \cite{norrie_quantum_2006}, but it is sufficient for modelling a  wide range of experiments nonetheless.} The resulting differential equation---the Gross--Pitaevskii equation---is nonlinear and using it to propagate a condensate wavefunction in time generally requires numerical techniques rather than analytic ones. My favourite numerical methods for doing so (which apply more generally to numerically evolving quantum systems of all kinds) are described in chapter \ref{chap:numerics}. In chapter \ref{chap:numerics} I also develop a variation on fourth-order Runge--Kutta integration which improves on one of its deficiencies for simulating quantum systems. I also present arguments that a fairly sophisticated method of discretising partial differential equations---the finite element discrete variable representation---may offer less computational efficiency that simpler methods for computing solutions of comparable accuracy to the Gross--Pitaevskii and Schr\"odinger wave equations.

As an experimental field, \textsc{bec} research involves the construction of apparatuses capable of implementing the techniques described in chapter \ref{chap:atomic_physics} in order to produce, control, and measure \textsc{bec}s. Chapter \ref{chap:experiment} describes some of the process of constructing such an apparatus, which involves a vacuum system, magnetic coils and optical systems. I present an optical layout for producing magneto-optically trapped $^{87}$Rb atoms (a step on the way to condensation) that I designed and assembled as an exchange student in the group of József Fortágh at the University of T\"ubingen's Physikalisches Institut.

Production, control, and measurement of cold atom systems require more than the necessary optics and magnetic sources to be installed---they must be controllable in a time-accurate way in order to execute the necessary cooling processes, manipulate the system as desired, and observe the results. Production of a condensate takes on the order of tens of seconds, requiring precisely timed pulses of laser light at specific frequencies, sweeps of magnetic field strengths, and frequency sweeps of radio and microwave radiation. This cannot all be done by human experimenters alone, and so requires computer automation of some kind. In chapter \ref{chap:software} I reproduce our publication on a suite of software programs, the \emph{labscript suite}. This software leverages modern software development techniques such as object orientation, abstraction and isolation as well as older principles---such as aspects of the Unix philosophy---to produce a powerful, maintainable, extensible system for designing, running and analysing shot-based experiments on commodity hardware.

As superfluids, \bec s have zero viscosity and as such can support persistent flows. In classical fluid dynamics the absence of viscosity means that a fluid cannot support vorticity,\footnote{This is because the motion of vorticity is described by a diffusion equation---with viscosity as the diffusion constant. When the diffusion constant is zero, there is no way for vorticity to enter the fluid from a boundary in the first place!} and must be irrotational. However, fluid circulation can still occur around points of zero fluid density, known as vortices. In \bec s this circulation is also quantised, in units of $h/m$.

These quantised vortices are topological defects---the phase of the macroscopic wavefunction winds by a multiple of $2\pi$ around them, and is undefined at the center of the vortex core itself.  Quantised vortices were observed in superfluid helium\footnote{In which 10\% or so of the atoms undergo Bose--Einstein condensation.} in the early 1960s \cite{vinen_detection_1961}, and in \bec\ in a dilute atomic gas in 1999 \cite{matthews_vortices_1999}. The formation, dynamics and decay of these vortices are believed to be important for the study of superfluid turbulence \cite{barenghi_quantized_2001}.

In chapter \ref{chap:velocimetry} I present simulations exploring the feasibility of imaging these vortices in-situ using \emph{tracer particles}. Atoms of one kind ($^{87}$Rb) may become trapped in the cores of quantised vortices in a condensate of another kind ($^{41}$K), and if imaged in a time-resolved way, reveal the motion of these vortices. A primary concern in any implementation of such a scheme is keeping the tracer atoms cold enough that they remain trapped in the vortex cores even as they scatter light for imaging. To that end, in chapter \ref{chap:velocimetry} I present modelling of a novel---if impractical---laser cooling scheme for Sisyphus cooling of $^{87}$Rb atoms in a $34\unit{G}$ magnetic field--- a field strength at which $^{87}$Rb and $^{41}$K repel each other strongly (leading to tighter trapping in the vortex cores).

In a \textsc{bec}, the wavelike behaviour of matter is apparent, unlike at higher temperatures at which atoms are well described as classical particles. Not only are atoms in a \textsc{bec} wavelike, they are described by a \emph{nonlinear} wave equation. As with nonlinear optical systems, they therefore exhibit wave-mixing behaviour whereby a number of momentum states may interact to produce additional momentum states. In chapter \ref{chap:wave_mixing} I describe our lab's four wave mixing experiment, which reproduced an existing result for \textsc{bec}s, and then our attempt at six wave mixing---a higher order effect (in the sense of perturbation theory). We did not observe the expected six wave mixing, rather we saw \emph{four} wave mixing despite the required resonance condition apparently being violated. This result agreed with mean-field theory  simulations however, implying that the result was unlikely to be due to some experimental error.

Atoms have spin, and when other spin-projection states cannot be disregarded, mean-field theory for spinful \textsc{bec}s takes the form of multiple, interacting fields. This allows for richer nonlinear dynamics that a single-component \textsc{bec}, with wave mixing producing new momentum states and new spin-projection states in tandem. In chapter \ref{chap:wave_mixing} I present simulation results showing this ``spin wave" mixing, although the main result is that---for $^{87}$Rb at least---the effect is very small and unlikely to be experimentally observable in existing $^{87}$Rb \textsc{bec} experiments.

As mentioned above, at high temperatures (higher than that at which atoms Bose-condense) atoms are well described as classical particles. This is true in the sense that the wavelike nature of the atoms can be disregarded---they move through space like classical billiard balls obeying Newtonian mechanics. The internal state of the atoms, however---for example the state of an outer shell electron---may not be well modelled by classical mechanics. Even at room temperature, an electron is poorly described as a classical charged particle orbiting a nucleus. When there is \emph{coupling} then, between this internal state of an atom and its motional state, the quantum-ness of the internal state can in some sense `leak' into its motional state even if the motion is otherwise modelled well classically. The classic example of this is the Stern--Gerlach experiment \cite{gerlach_experimentelle_1922}, in which a beam of atoms splits into two beams as it passes through a magnetic field gradient. A similar situation arises for atoms in a magnetic trap---a common feature of cold atom experiments and often used in the final stage of cooling to \textsc{bec}. To correctly model the losses of atoms from these traps, one needs to model the internal state of the atoms quantum-mechanically, but it is computationally expensive to also model their spatial motion using full quantum wavefunctions. We would like a way to model the atoms' motion classically, but in such a way that it can reproduce Stern--Gerlach separation---with modelled atoms taking one or the other trajectory probabilistically, with the probabilities consistent with those of a fully quantum treatment. In chapter \ref{chap:hvsc} I present such a model, one that is based on a \emph{hidden variable} carried around with each atom being modelled, which selects one of the atom's internal eigenstates. The apparent definiteness of the hidden variable allows the spatial motion part of the modelling to treat the atoms' spin projection degree of freedom as if it were in a definite state, allowing the modelling to take a single, definite trajectory. The hidden variable itself is evolved using a stochastic hidden variable theory that ensures its probability of corresponding to any particular spin-projection state is consistent with with the underlying quantum evolution of the atom's internal degrees of freedom.